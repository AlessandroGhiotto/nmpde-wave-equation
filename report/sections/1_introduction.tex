\section{Introduction}
The wave equation is one of the most important models in mathematical physics: it describes how disturbances propagate through a medium and appears in acoustics, electromagnetism, seismology and structural dynamics \cite{evans2010partial}. Being a second-order hyperbolic PDE, it relates the acceleration of the solution to its spatial curvature. Solving it numerically in an accurate and efficient way is therefore of great practical interest.

Beyond the classical stability constraint given by the Courant--Friedrichs--Lewy (CFL) condition \cite{courant1928CFL}, numerical methods for the wave equation can suffer from two additional artefacts \cite{hughes2012finite}: \emph{numerical dispersion}, which alters the propagation speed of different frequency components and distorts the wave shape, and \emph{numerical dissipation}, which causes a non-physical decay of the amplitude. Controlling these effects requires a careful choice of both the spatial mesh and the time-integration scheme.

In this work we compare several well-known time-stepping methods \cite{newmark1959method, quarteroni2008numerical} applied to the finite-element semi-discretization of the wave equation. The goal is to go beyond a purely theoretical analysis and evaluate, through numerical experiments, the practical trade-offs between accuracy, energy conservation, and computational cost. In particular, we study convergence rates, dissipation and dispersion properties, and parallel scalability on the HPC cluster of Politecnico di Milano.

\begin{figure}[h]
    \centering
    \subfloat[Square wave]{\includegraphics[width=0.32\textwidth]{sol-square.png}}
    \hfill
    \subfloat[Standing wave]{\includegraphics[width=0.32\textwidth]{sol-standing.png}}
    \hfill
    \subfloat[Traveling wave]{\includegraphics[width=0.32\textwidth]{sol-traveling.png}}
    \caption{Snapshots of three representative simulations produced by our solver. Animated videos of these and other test cases are available in this \href{https://www.youtube.com/playlist?list=PLh9fwY76Oobay4rXdlAKkHKuC61a1lL61}{YouTube playlist}.}
    \label{fig:solution_examples}
\end{figure}

