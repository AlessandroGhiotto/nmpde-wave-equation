\section{Introduction}
The wave equation is a fundamental mathematical model used to describe the propagation of disturbances in various physical media, spanning from acoustics and electromagnetism to seismology and structural dynamics \cite{evans2010partial}. In its second-order form, this hyperbolic partial differential equation captures the essential physics of wave motion by balancing temporal acceleration with spatial curvature. Due to its wide range of applications, developing accurate and efficient numerical solvers is crucial for predicting complex dynamic behaviors.

The numerical simulation of wave phenomena, however, introduces significant challenges that go beyond simple stability. While the Courant-Friedrichs-Lewy (CFL) condition \cite{courant1928CFL} provides a standard limit for time-stepping in explicit schemes, the long-term accuracy of the solution is often compromised by numerical dispersion and dissipation \cite{hughes2012finite}. Numerical dispersion leads to frequency-dependent velocity errors that distort the wave shape, while numerical dissipation causes an unphysical decay of the wave's amplitude. Mitigating these errors demands a careful balance between spatial resolution and the choice of the time-integration strategy.

Motivated by these challenges, this work conducts a systematic evaluation of established time-marching strategies \cite{newmark1959method, quarteroni2008numerical} applied to the semi-discrete wave equation. The primary objective is to move beyond theoretical stability bounds and provide a clear characterization of the practical trade-offs between computational efficiency and physical fidelity in large-scale scenarios. By examining how different algorithmic choices impact energy conservation, phase accuracy, and parallel execution times, this study aims to establish robust guidelines for dynamic wave simulations.

