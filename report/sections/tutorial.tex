%!TeX root = ../report.tex
\section{Figures, Tables and Algorithms}

Figures, Tables and Algorithms have to contain a Caption that describes their content, and have to be properly referred in the text.

\subsection{Figures}
\label{subsec:figures}

For including pictures in your text you can use \texttt{TikZ} for high-quality hand-made figures \cite{tikz},
or just include them with the command
\begin{verbatim}
\includegraphics[options]{filename.xxx}
\end{verbatim}
Here xxx is the correct format, e.g.  \verb|.png|, \verb|.jpg|, \verb|.eps|, \dots.

\begin{figure}[H]
    \centering
    \includegraphics[width=0.3\textwidth]{images/logo_polimi_scritta.eps}
    \caption{Caption of the Figure.}
    \label{fig:quadtree}
\end{figure}

Thanks to the \texttt{\textbackslash subfloat} command, a single figure, such as Figure~\ref{fig:quadtree},
can contain multiple sub-figures with their own caption and label, e.g. Figure~\ref{fig:polimi_logo1} and Figure~\ref{fig:polimi_logo2}. 

\begin{figure}[H]
    \centering
    \subfloat[One PoliMi logo.\label{fig:polimi_logo1}]{
        \includegraphics[scale=0.5]{images/logo_polimi_scritta.eps}
    }
    \quad
    \subfloat[Another one PoliMi logo.\label{fig:polimi_logo2}]{
        \includegraphics[scale=0.5]{images/logo_polimi_scritta2.eps}
    }
    \caption[]{Caption of the Figure.}
    \label{fig:quadtree2}
\end{figure}

\subsection{Tables}
\label{subsec:tables}

You can also consider to highlight selected columns or rows in order to make tables more readable.
Moreover, with the use of \texttt{table*} and the option \texttt{bp} it is possible to align them at the bottom of the page. One example is presented in Table~\ref{table:exampleC}. 

\begin{table*}[bp]
\centering 
    \begin{tabular}{|p{3em} | c | c | c | c | c | c|}
    \hline
%    \rowcolor{bluePoli!40}
     & \textbf{column1} & \textbf{column2} & \textbf{column3} & \textbf{column4} & \textbf{column5} & \textbf{column6} \T\B \\
    \hline \hline
    \textbf{row1} & 1 & 2 & 3 & 4 & 5 & 6 \T\B\\
    \textbf{row2} & a & b & c & d & e & f \T\B\\
    \textbf{row3} & $\alpha$ & $\beta$ & $\gamma$ & $\delta$ & $\phi$ & $\omega$ \T\B\\
    \textbf{row4} & alpha & beta & gamma & delta & phi & omega \B\\
    \hline
    \end{tabular}
    \\[10pt]
    \caption{Highlighting the columns}
    \label{table:exampleC}
\end{table*}

\subsection{Algorithms}
\label{subsec:algorithms}

Pseudo-algorithms can be written in \LaTeX{} with the \texttt{algorithm} and \texttt{algorithmic} packages.
An example is shown in Algorithm~\ref{alg:var}.
\begin{algorithm}[H]
\label{alg:example}
\caption{Name of the Algorithm}
\label{alg:var}
\label{protocol1}
\begin{algorithmic}[1]
\STATE Initial instructions
\FOR{$for-condition$}
\STATE{Some instructions}
\IF{$if-condition$}
\STATE{Some other instructions}
\ENDIF
\ENDFOR
\WHILE{$while-condition$}
\STATE{Some further instructions}
\ENDWHILE
\STATE Final instructions
\end{algorithmic}
\end{algorithm} 