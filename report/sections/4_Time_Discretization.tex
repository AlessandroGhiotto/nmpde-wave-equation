
\section{Time Discretization}
\label{sec:time_discretization}

The spatial discretization detailed in (9) yields a semi-discrete dynamical system subject to  the initial conditions $\mathbf{U}(0) = \mathbf{U}_0$ and $\frac{\partial \mathbf{U}}{\partial t}(0) = \mathbf{V}_0$. 

To apply standard time-marching techniques, it is common practice to recast (9) as a system of first-order ordinary differential equations \cite{quarteroni2008numerical}. We introduce the discrete velocity vector $\mathbf{V}(t) = \frac{\partial \mathbf{U}}{\partial t}(t)$, allowing us to rewrite the problem as two coupled equations:
\begin{equation}
\label{eq:first_order_system}
\begin{cases}
\frac{\partial \mathbf{U}}{\partial t}(t) - \mathbf{V}(t) = \mathbf{0} \\
\mathbf{M}\frac{\partial \mathbf{V}}{\partial t}(t) + \mathbf{A}\mathbf{U}(t) = \mathbf{G}(t)
\end{cases}
\end{equation}

Boundary conditions are naturally inherited by the velocity field; for a prescribed Dirichlet datum $\mathbf{U}(t) = \phi(t)$ on the boundary nodes, the corresponding velocity constraint is strictly enforced as $\mathbf{V}(t) = \frac{\partial \phi}{\partial t}(t)$. This coupled first-order system forms the basis for the $\theta$-method family, while the original second-order form (9) is utilized directly by the Newmark-$\beta$ integration scheme.

\subsection{The $\theta$-Method}
\label{subsec:theta_method}

To discretize the first-order system \eqref{eq:first_order_system}, we adopt a family of implicit-explicit schemes parameterized by $\theta \in [0,1]$     \cite{quarteroni2008numerical}. Time derivatives are approximated using finite differences over a time step $\Delta t$, and the equation terms are evaluated as a convex combination of levels $t^n$ and $t^{n+1}$. The discretized system reads:

\begin{equation}
\label{eq:theta_u}
\frac{\mathbf{U}^{n+1} - \mathbf{U}^n}{\Delta t} = \theta \mathbf{V}^{n+1} + (1-\theta) \mathbf{V}^n
\end{equation}
\begin{equation}
\label{eq:theta_v}
\mathbf{M} \frac{\mathbf{V}^{n+1} - \mathbf{V}^n}{\Delta t} + \mathbf{A} (\theta \mathbf{U}^{n+1} + (1-\theta) \mathbf{U}^n) = \mathbf{G}_{\theta}^{n+1}
\end{equation}
where $\mathbf{G}_{\theta}^{n+1} = \theta \mathbf{G}^{n+1} + (1-\theta) \mathbf{G}^n$. 

By isolating $\mathbf{V}^{n+1}$ in \eqref{eq:theta_u} and substituting it into \eqref{eq:theta_v}, we obtain a single linear system for the unknown displacement $\mathbf{U}^{n+1}$:
\begin{equation}
\label{eq:theta_final}
(\mathbf{M} + \theta^2 \Delta t^2 \mathbf{A}) \mathbf{U}^{n+1} = \mathbf{M}(\mathbf{U}^n + \Delta t \mathbf{V}^n) - \theta(1-\theta)\Delta t^2 \mathbf{A} \mathbf{U}^n + \theta \Delta t^2 \mathbf{G}_{\theta}^{n+1}
\end{equation}

At each time step, we solve \eqref{eq:theta_final} for $\mathbf{U}^{n+1}$. Subsequently, the velocity $\mathbf{V}^{n+1}$ is explicitly updated using \eqref{eq:theta_u}. The parameter $\theta$ dictates the properties of the scheme:
\begin{itemize}
    \item $\theta = 0$: \textbf{Forward Euler}. Fully explicit and conditionally stable.
    \item $\theta = 1/2$: \textbf{Crank-Nicolson}. Implicit, second-order accurate in time, and non-dissipative.
    \item $\theta = 1$: \textbf{Backward Euler}. Fully implicit, generally unconditionally stable, but highly dissipative.
\end{itemize}


\subsection{Newmark Time Integration}
\label{subsec:newmark}

The Newmark-$\beta$ method is a widely used time-marching scheme for second-order dynamical systems \cite{newmark1959method, hughes2012finite}. It computes the state at $t^{n+1} = t^n + \Delta t$ by introducing two parameters, $\beta$ and $\gamma$, which dictate the stability, accuracy, and numerical dissipation.

The discrete displacement $\mathbf{U}$ and velocity $\mathbf{V}$ are updated using the acceleration $\mathbf{a} \approx \frac{\partial^2 \mathbf{U}}{\partial t^2}$:
\begin{align}
\label{eq:newmark_u}
\mathbf{U}^{n+1} &= \mathbf{U}^n + \Delta t \mathbf{V}^n + \Delta t^2 \left[ \left(\frac{1}{2} - \beta\right)\mathbf{a}^n + \beta \mathbf{a}^{n+1} \right] \\
\label{eq:newmark_v}
\mathbf{V}^{n+1} &= \mathbf{V}^n + \Delta t \left[ (1 - \gamma)\mathbf{a}^n + \gamma \mathbf{a}^{n+1} \right]
\end{align}

To compute the unknown acceleration $\mathbf{a}^{n+1}$, we enforce dynamic equilibrium at $t^{n+1}$:
\begin{equation}
\mathbf{M}\mathbf{a}^{n+1} + \mathbf{A}\mathbf{U}^{n+1} = \mathbf{G}^{n+1}
\end{equation}

Substituting \eqref{eq:newmark_u} into this equilibrium equation yields a linear system for $\mathbf{a}^{n+1}$:
\begin{equation}
\label{eq:newmark_solve}
(\mathbf{M} + \beta \Delta t^2 \mathbf{A})\mathbf{a}^{n+1} = \mathbf{G}^{n+1} - \mathbf{A}\mathbf{U}^n - \Delta t \mathbf{A}\mathbf{V}^n - \Delta t^2 \left(\frac{1}{2} - \beta\right)\mathbf{A}\mathbf{a}^n
\end{equation}

At each time step, we solve \eqref{eq:newmark_solve} for $\mathbf{a}^{n+1}$, and then explicitly update $\mathbf{U}^{n+1}$ and $\mathbf{V}^{n+1}$. The method's behavior depends heavily on the chosen parameters:
\begin{itemize}
    \item $\gamma = 1/2, \beta = 1/4$: \textbf{Average Acceleration Method}. Unconditionally stable and non-dissipative.
    \item $\beta > 1/4$: Introduces numerical dissipation, useful for damping high-frequency modes.
    \item $\gamma < 1/2$: Explicit scheme (conditionally stable).
\end{itemize}


