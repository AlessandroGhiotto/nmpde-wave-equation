
\section{Time Discretization}
\label{sec:time_discretization}

The spatial discretization detailed in \eqref{eq:semi_discrete_system} yields a semi-discrete dynamical system subject to the initial conditions $\mathbf{u}(0) = \mathbf{u}_0$ and $\partial_t \mathbf{u}(0) = \mathbf{v}_0$. 

A standard approach \cite{quarteroni2008numerical} is to rewrite \eqref{eq:semi_discrete_system} as a first-order system by introducing the velocity vector $\mathbf{v}(t) = \partial_t \mathbf{u}(t)$:
\begin{equation}
    \label{eq:first_order_system}
    \begin{cases}
    \partial_t \mathbf{u}(t) - \mathbf{v}(t) = \mathbf{0} \\
    \mathbf{M}\,\partial_t \mathbf{v}(t) + \mathbf{A}\mathbf{u}(t) = \mathbf{G}(t) \\
    \mathbf{u}(0) = \mathbf{u}_0 \\
    \mathbf{v}(0) = \mathbf{v}_0
    \end{cases}
\end{equation}

% On the boundary nodes, if $\mathbf{u}(t) = \phi(t)$ then $\mathbf{v}(t) = \partial_t \phi(t)$. 
The $\theta$-method is applied to this first-order system, while the Newmark-$\beta$ method works directly on the second-order form~\eqref{eq:semi_discrete_system}.

\subsection{The $\theta$-Method}
\label{subsec:theta_method}

We discretize \eqref{eq:first_order_system} with a one-parameter family of schemes controlled by $\theta \in [0,1]$ \cite{quarteroni2008numerical}. The time derivatives are replaced by finite differences over a step $\Delta t>0$, and the right-hand side terms are weighted between $t^n$ and $t^{n+1}$, where $t^k = k \Delta t$:

\begin{equation}
    \label{eq:theta_u}
    \frac{\mathbf{u}^{n+1} - \mathbf{u}^n}{\Delta t} = \theta\, \mathbf{v}^{n+1} + (1-\theta) \mathbf{v}^n
\end{equation}

\begin{equation}
    \label{eq:theta_v}
    \mathbf{M}\, \frac{\mathbf{v}^{n+1} - \mathbf{v}^n}{\Delta t} + \mathbf{A} (\theta\, \mathbf{u}^{n+1} + (1-\theta) \mathbf{u}^n) = \mathbf{G}_{\theta}^{n+1}
\end{equation}

where $\mathbf{G}_{\theta}^{n+1} = \theta\, \mathbf{G}^{n+1} + (1-\theta)\mathbf{G}^n$. 

We first rearrange \eqref{eq:theta_v} to isolate $\mathbf{M}\,\mathbf{v}^{n+1}$:
\begin{equation}
    \label{eq:theta_v_rearranged}
    \mathbf{M}\,\mathbf{v}^{n+1} = \mathbf{M}\,\mathbf{v}^n - \Delta t\, \mathbf{A}\bigl[\theta\, \mathbf{u}^{n+1} + (1-\theta)\, \mathbf{u}^n\bigr] + \Delta t\, \mathbf{G}_{\theta}^{n+1}
\end{equation}

Then we multiply \eqref{eq:theta_u} by $\mathbf{M}$ and substitute \eqref{eq:theta_v_rearranged} into it, obtaining a linear system for $\mathbf{u}^{n+1}$:

\begin{equation}
    \label{eq:theta_final}
    (\mathbf{M} + \theta^2 \Delta t^2 \mathbf{A})\, \mathbf{u}^{n+1} = \mathbf{M}(\mathbf{u}^n + \Delta t\, \mathbf{v}^n) - \Delta t^2\, \theta(1-\theta) \mathbf{A}\, \mathbf{u}^n +  \Delta t^2\, \theta\, \mathbf{G}_{\theta}^{n+1}
\end{equation}

At each time step we solve \emph{two} SPD linear systems sequentially:
\begin{enumerate}
    \item Solve \eqref{eq:theta_final} for $\mathbf{u}^{n+1}$ (the right-hand side depends only on quantities at $t^n$).
    \item Solve \eqref{eq:theta_v_rearranged} for $\mathbf{v}^{n+1}$ (using the just-computed $\mathbf{u}^{n+1}$).
\end{enumerate}

\noindent The value of $\theta$ determines the character of the scheme:
\begin{itemize}
    \item $\theta = 0$: \textbf{Forward Euler}. Fully explicit, requires $\Delta t$ small enough for stability.
    \item $\theta = 1/2$: \textbf{Crank-Nicolson}. Implicit, second-order accurate in time, and non-dissipative.
    \item $\theta = 1$: \textbf{Backward Euler}. Fully implicit, unconditionally stable, but highly dissipative.
\end{itemize}


\subsection{Newmark Time Integration}
\label{subsec:newmark}

The Newmark-$\beta$ method \cite{newmark1959method, hughes2012finite} is designed for second-order systems and advances the solution from $t^n$ to $t^{n+1} = t^n + \Delta t$ using two parameters $\beta$ and $\gamma$ that control stability, accuracy, and dissipation.

The discrete displacement $\mathbf{u}$ and velocity $\mathbf{v}$ are updated using the acceleration $\mathbf{a} \approx \partial_{tt} \mathbf{u}$:
\begin{align}
    \label{eq:newmark_u}
    \mathbf{u}^{n+1} &= \mathbf{u}^n + \Delta t \mathbf{v}^n + \Delta t^2 \left[ \beta \mathbf{a}^{n+1} + \left(\frac{1}{2} - \beta\right)\mathbf{a}^n \right] \\
    \label{eq:newmark_v}
    \mathbf{v}^{n+1} &= \mathbf{v}^n + \Delta t \left[ \gamma \mathbf{a}^{n+1} + (1 - \gamma)\mathbf{a}^n \right]
\end{align}

The unknown $\mathbf{a}^{n+1}$ is found by requiring that the semi-discrete equation holds at $t^{n+1}$:
\begin{equation}
    \mathbf{M}\mathbf{a}^{n+1} + \mathbf{A}\mathbf{u}^{n+1} = \mathbf{G}^{n+1}
\end{equation}

Substituting \eqref{eq:newmark_u} into this equilibrium equation yields a linear system for $\mathbf{a}^{n+1}$:
\begin{equation}
    \label{eq:newmark_solve}
    (\mathbf{M} + \Delta t^2 \beta \mathbf{A})\mathbf{a}^{n+1} = \mathbf{G}^{n+1} - \mathbf{A}\mathbf{u}^n - \Delta t \mathbf{A}\mathbf{v}^n - \Delta t^2 \left(\frac{1}{2} - \beta\right)\mathbf{A}\mathbf{a}^n
\end{equation}

At each step we solve \eqref{eq:newmark_solve} for $\mathbf{a}^{n+1}$ and then update $\mathbf{u}^{n+1}$ and $\mathbf{v}^{n+1}$ algebraically. Notable parameter choices are:
\begin{itemize}
    \item $\gamma = 1/2, \beta = 1/4$: \textbf{Average Acceleration Method}. Unconditionally stable and non-dissipative.
    \item $\gamma = 1/2, \beta = 0$: \textbf{Central Difference Method}. Explicit and conditionally stable.
    \item $\gamma > 1/2$ (with $\beta \geq \tfrac{1}{4}(\gamma + \tfrac{1}{2})^2$): Introduces controlled numerical dissipation, useful for damping spurious high-frequency oscillations.
\end{itemize}


