
\section{Continuous Wave Problem}
\label{sec:continuous_problem}

\subsection{Strong Formulation}
The propagation of waves in a bounded domain $\Omega \subset \mathbb{R}^2$ ,as preannounced, is modeled by a second-order hyperbolic Partial Differential Equation \cite{salsa2015partial} . We seek the displacement field $u(\mathbf{x},t)$ that satisfies the following initial-boundary value problem :

\begin{equation}
\label{eq:strong_form}
\begin{cases} 
\frac{\partial^2 u}{\partial t^2} - \nabla \cdot (c^2 \nabla u) = f(\mathbf{x},t) & \text{in } \Omega \times (0, T] \\
u(\mathbf{x}, t) = g(\mathbf{x}, t) & \text{on } \partial \Omega \times [0, T] \\
u(\mathbf{x}, 0) = u_0(\mathbf{x}) & \text{in } \Omega \\
\frac{\partial u}{\partial t}(\mathbf{x}, 0) = v_0(\mathbf{x}) & \text{in } \Omega 
\end{cases}
\end{equation}

In this framework, $c(\mathbf{x}, t)$ denotes the wave propagation speed, which is handled in our implementation as a space-time dependent function . The term $f$ represents the external forcing , while $u_0$ and $v_0$ define the initial displacement and velocity, respectively. The Dirichlet boundary condition $g$ constrains the solution on the boundary $\partial \Omega$ throughout the simulation interval $[0, T]$.

\subsection{Weak Formulation}

To apply the Finite Element Method (FEM), we derive the variational form of the strong problem. We begin by defining the appropriate functional spaces for the solution and the test functions. The test space is defined as $V_0 = H^1_0(\Omega) = \{v \in H^1(\Omega) : v|_{\partial\Omega} = 0\}$, which consists of functions with zero trace on the boundary. In the presence of non-homogeneous Dirichlet conditions $g \neq 0$, the solution $u$ belongs to the trial space $V_g = \{v \in H^1(\Omega) : v|_{\partial\Omega} = g\}$.

\vspace{0.5cm}
To ensure compatibility with the Lax-Milgram Lemma \cite{quarteroni2008numerical}, which requires trial and test functions to belong to the same Hilbert space, we introduce a \textbf{Lifting Operator} $\mathcal{R}$. Let $u_g = \mathcal{R}g \in H^1(\Omega)$ be a function such that its trace on the boundary $\partial \Omega$ is exactly $g$. We decompose the global solution as:
\begin{equation}
    u(\mathbf{x}, t) = u_0(\mathbf{x}, t) + u_g(\mathbf{x}, t)
\end{equation}
where $u_0 \in V_0$ is the new unknown function with zero trace on the boundary.

The derivation proceeds by multiplying the strong equation by a test function $v \in V_0$ and integrating over the domain $\Omega$:
\begin{equation}
    \int_{\Omega} \frac{\partial^2 u}{\partial t^2} v \, d\Omega - \int_{\Omega} \nabla \cdot (c^2 \nabla u) v \, d\Omega = \int_{\Omega} f v \, d\Omega
\end{equation}
Applying Green's first identity to the second-order spatial term, we obtain:
\begin{equation}
    \int_{\Omega} \frac{\partial^2 u}{\partial t^2} v \, d\Omega + \int_{\Omega} c^2 \nabla u \cdot \nabla v \, d\Omega - \int_{\partial \Omega} (c^2 \nabla u \cdot \mathbf{n}) v \, d\gamma = \int_{\Omega} f v \, d\Omega
\end{equation}
Since $v = 0$ on $\partial \Omega$, the boundary integral vanishes. Substituting the decomposition $u = u_0 + u_g$ and exploiting the linearity of the operators, the semi-discrete weak formulation is expressed in the following way: 

\begin{equation}
\label{eq:weak_problem_lifting}
\begin{cases}
\forall t \in (0,T), \text{ find } u_0(t) \in V_0 \text{ such that:} \\
\left( \frac{\partial^2 u_0}{\partial t^2}, v \right)_{L^2(\Omega)} + a(u_0, v) = F(v) - \left( \frac{\partial^2 u_g}{\partial t^2}, v \right)_{L^2(\Omega)} - a(u_g, v) \quad \forall v \in V_0
\end{cases}
\end{equation}
with the initial conditions: 

$$
u(x,0)=u_0(x)
$$

$$
\frac{\partial u}{\partial t}(x,0)=u_1(x)
$$

This formulation eliminates the second derivatives in space, making the problem compatible with finite $H^1$ spaces and allowing numerical discretization. Moreover by homogenizing the problem through the lifting operator, the bilinear form
a(·, ·) remains continuous and coercive on the Hilbert space $H^1_0(\Omega)$, guaranteeing
the existence and uniqueness of the solution \cite{evans2010partial, quarteroni2008numerical}.
\vspace{0.5cm}

The bilinear and linear forms utilized in the implementation are defined as:
\begin{itemize}
    \item \textbf{Mass form}: $(w, v)_{L^2(\Omega)} = \int_{\Omega} w v \, d\Omega$(used for assembling the mass matrix $\mathbf{M}$).
    \item \textbf{Stiffness form}: $a(u, v) = \int_{\Omega} c^2 \nabla u \cdot \nabla v \, d\Omega$(used for assembling the stiffness matrix $\mathbf{A}$).
    \item \textbf{Forcing form}: $F(v) = \int_{\Omega} f v \, d\Omega$, representing the external source term.
\end{itemize}
