
\section{Continuous Wave Problem}
\label{sec:continuous_problem}

\subsection{Strong Formulation}
We consider a bounded domain $\Omega \subset \mathbb{R}^2$. The wave propagation is governed by the following second-order hyperbolic PDE \cite{salsa2015partial}: find $u(\mathbf{x},t)$ such that

\begin{equation}
    \label{eq:strong_form}
    \begin{cases} 
    \partial_{tt} u(\mathbf{x}, t) - \nabla \cdot (c(\mathbf{x}, t)^2\, \nabla u(\mathbf{x}, t)) = f(\mathbf{x},t) & \text{in } \Omega \times (0, T] \\
    u(\mathbf{x}, t) = \phi(\mathbf{x}, t) & \text{on } \partial \Omega \times [0, T] \\
    u(\mathbf{x}, 0) = u_0(\mathbf{x}) & \text{in } \Omega \\
    \partial_t u(\mathbf{x}, 0) = v_0(\mathbf{x}) & \text{in } \Omega 
    \end{cases}
\end{equation}

Here $c$ represents the wave speed (eventually depending on space and time), $f$ is an external forcing term, $u_0$ and $v_0$ are the initial displacement and velocity, and $\phi$ prescribes the Dirichlet boundary condition on $\partial \Omega$ for all $t \in [0, T]$.

\subsection{Weak Formulation}

In order to apply the Finite Element Method (FEM), we need to write the weak formulation of the problem.\\
We define the test space $V = H^1_0(\Omega) = \{v \in H^1(\Omega) : v = 0 \text{ in } \partial\Omega\}$, which consists of functions wich are zero on the boundary, whereas the solution $u$ lives in the space $V_\phi = \{v \in H^1(\Omega) : v = \phi \text{ in } \partial\Omega\}$.

\vspace{0.5cm}
To ensure compatibility with the Lax-Milgram Lemma \cite{quarteroni2008numerical}, we introduce a \textbf{lifting operator} $\mathcal{R}\phi \in V_\phi$. We then split the solution as
\begin{equation}
    \label{eq:solutionsplit}
    u(\mathbf{x}, t) = \hat{u}(\mathbf{x}, t) + \mathcal{R}\phi(\mathbf{x}, t),
\end{equation}
where $\hat{u} \in V$ is the new unknown, which has homogeneous Dirichlet boundary conditions.

We multiply the PDE by a test function $v \in V$ and integrate over $\Omega$:
\begin{equation}
    \int_{\Omega} \partial_{tt} u \; v \, d\Omega - \int_{\Omega} \nabla \cdot (c^2 \nabla u) v \, d\Omega = \int_{\Omega} f v \, d\Omega, \qquad \forall v \in V.
\end{equation}
Applying Green's formula to the diffusion term gives
\begin{equation}
    \int_{\Omega} \partial_{tt} u \; v \, d\Omega + \int_{\Omega} c^2 \nabla u \cdot \nabla v \, d\Omega - \int_{\partial \Omega} (c^2 \nabla u \cdot \mathbf{n}) v \, d\gamma = \int_{\Omega} f v \, d\Omega, \qquad \forall v \in V.
\end{equation}
Because $v = 0$ on $\partial \Omega$, the boundary integral vanishes. Substituting \eqref{eq:solutionsplit}and using linearity, we get the weak formulation: 

\begin{equation}
\label{eq:weak_problem_lifting}
\begin{cases}
\forall t \in (0,T), \text{ find } \hat{u}(t) \in V \text{ such that:} \\
\int_{\Omega} \partial_{tt} \hat{u} \; v \, d\Omega + a(\hat{u}, v) = G(v) \qquad \forall v \in V \\
\hat{u}(\mathbf{x},0) = u_0(\mathbf{x}) - \mathcal{R}\phi(\mathbf{x},0) \\
\partial_t \hat{u}(\mathbf{x},0) = v_0(\mathbf{x}) - \partial_t \mathcal{R}\phi(\mathbf{x},0)
\end{cases}
\end{equation}
% with the initial conditions: 
% $$
% u(x,0)=u_0(x)
% $$
% $$
% \frac{\partial u}{\partial t}(x,0)=u_1(x)
% $$

The bilinear and linear forms are defined as follows:
\begin{itemize}
    % \item \textbf{Mass form}: $m(w, v) = \int_{\Omega} w v \, d\Omega$ 
    % (used for assembling the mass matrix $\mathbf{M}$).
    \item \textbf{Stiffness form}:  Let $a: V\times V \rightarrow \mathbb{R}$, with $a(u, v) = \int_{\Omega} c^2 \nabla u \cdot \nabla v \, d\Omega$ 
    % (used for assembling the stiffness matrix $\mathbf{A}$).
    \item \textbf{RHS}: Let $G: V\rightarrow \mathbb{R}$, with $G(v) = \int_{\Omega} f v \, d\Omega - \int_{\Omega} \partial_{tt} \mathcal{R}\phi \; v \, d\Omega - a(\mathcal{R}\phi, v) $, which gathers all the terms that do not involve the unknown $\hat{u}$.
\end{itemize}


The integration by parts removes the second-order spatial derivatives, so the formulation only requires $H^1$ regularity. Because of the lifting, the unknown $\hat{u}$ satisfies homogeneous Dirichlet conditions, and the bilinear form
$a(\cdot, \cdot)$ is continuous and coercive on $H^1_0(\Omega)$, which guarantees existence and uniqueness of the solution \cite{evans2010partial, quarteroni2008numerical}.
