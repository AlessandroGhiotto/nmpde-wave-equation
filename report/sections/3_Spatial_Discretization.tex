
\section{Spatial Discretization}
\label{sec:spatial_discretization}

To discretize the problem in space using the standard Galerkin method, we introduce a finite-dimensional subspace $V_{h} \subset H^1_0(\Omega)$, where $\dim(V_{h}) = N_h < +\infty$. In our implementation, this space is constructed using simplicial finite elements of polynomial degree $r$, specifically through the \texttt{FE\_SimplexP} class provided by the \textbf{deal.II} library \cite{arndt2021dealii}. The goal is to find an approximation $u_h(t) \in V_{h}$ of the exact solution by imposing the weak formulation within the discrete space, leading to the following numerical problem:

\begin{equation}
\label{eq:discrete_weak_form}
\begin{cases}
\forall t \in (0,T), \text{ find } u_h(t) \in V_{h} \text{ such that:} \\
\left( \frac{\partial^2 u_h}{\partial t^2}, v_h \right)_{L^2(\Omega)} + a(u_h, v_h) = G(v_h) \quad \forall v_h \in V_{h}
\end{cases}
\end{equation}

To construct an algebraic representation, we consider a basis of $V_{h}$ denoted by the shape functions $\{\varphi_j\}_{j=1}^{N_h}$. The discrete solution is expanded as:
\begin{equation}
u_h(\mathbf{x},t) = \sum_{j=1}^{N_h} U_j(t) \varphi_j(\mathbf{x})
\end{equation}
where the coefficients $U_j(t)$ represent the degrees of freedom (DoFs) of the system, collected in the vector $\mathbf{U}(t) = [U_1(t), U_2(t), \dots, U_{N_h}(t)]^T$ . Substituting this expansion into the weak formulation and choosing $v_h = \varphi_i$ as test functions, the linearity of the integral yields:
\begin{equation}
\sum_{j=1}^{N_h} \frac{d^2 U_j(t)}{dt^2} \int_{\Omega} \varphi_j \varphi_i \, d\Omega + \sum_{j=1}^{N_h} U_j(t) \int_{\Omega} c^2 \nabla \varphi_j \cdot \nabla \varphi_i \, d\Omega = \mathbf{G}_i(t) \quad i=1,..., N_h
\end{equation}

This allows us to identify the global matrices assembled by the solver:
\begin{itemize}
    \item \textbf{Mass Matrix} $\mathbf{M}$: $M_{ij} = \int_{\Omega} \varphi_j \varphi_i \, d\Omega$.
    \item \textbf{Stiffness Matrix} $\mathbf{A}$: $A_{ij} = \int_{\Omega} c^2 \nabla \varphi_j \cdot \nabla \varphi_i \, d\Omega$.
    \item \textbf{RHS} $\mathbf{G}(t) = [G_1(t), G_2(t), \dots, G_{N_h}(t)]^T$.
\end{itemize}

% The assembly process is performed in parallel using \textbf{Trilinos} distributed matrices. For each active cell in the mesh, local contributions are computed using numerical quadrature, specifically the \texttt{QGaussSimplex} rule of order $r+1$ \cite{quarteroni2008numerical}. 

The resulting semi-discrete dynamical system is:
\begin{equation}
\label{eq:semi_discrete_system}
\mathbf{M} \, \frac{\partial^2 \mathbf{U}}{\partial t^2}(t) + \mathbf{A} \mathbf{U}(t) = \mathbf{G}(t)
\end{equation}

This second-order system of Ordinary Differential Equations represents the spatial discretization (semi-discrete) of the wave equation. 

%In the presence of non-homogeneous Dirichlet conditions, the solver modifies this system during the solution phase to ensure that the degrees of freedom on the boundary correctly reflect the prescribed data $g(\mathbf{x},t)$.

