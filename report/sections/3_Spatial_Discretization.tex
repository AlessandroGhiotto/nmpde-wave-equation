
\section{Spatial Discretization}
\label{sec:spatial_discretization}

We apply the standard Galerkin method by introducing a finite-dimensional subspace $V_{h} \subset V$ such that $\dim(V_{h}) = N_h < +\infty$.\\The semi-discretized problem that we obtain is the following:

\begin{equation}
    \label{eq:discrete_weak_form}
    \begin{cases}
    \forall t \in (0,T), \text{ find } u_h(t) \in V_{h} \text{ such that:} \\
    \int_{\Omega} \partial_{tt} u_h \; v_h \, d\Omega + a(u_h, v_h) = G(v_h) \qquad \forall v_h \in V_{h} \\
    u_h(\mathbf{x},0) = u_{0,h}(\mathbf{x}) \\
    \partial_t u_h(\mathbf{x},0) = v_{0,h}(\mathbf{x})
    \end{cases}
\end{equation}

\noindent where $u_{0,h}$ and $v_{0,h}$ are the projections onto $V_h$ of the continuous initial data $u_0 - \mathcal{R}\phi(\cdot,0)$ and $v_0 - \partial_t \mathcal{R}\phi(\cdot,0)$, respectively.

Let $\{\varphi_j\}_{j=1}^{N_h}$ be a basis of shape functions for the space $V_{h}$. We expand the discrete solution as
\begin{equation}
    u_h(\mathbf{x},t) = \sum_{j=1}^{N_h} u_j(t) \varphi_j(\mathbf{x}),
\end{equation}

where the coefficients $u_j(t)$ are the degrees of freedom (DoFs). Plugging this into the weak form and using as test function $v$ the shape functions $\varphi_i$ gives
\begin{equation}
    \sum_{j=1}^{N_h} \frac{d^2 u_j}{dt^2}(t) \int_{\Omega} \varphi_j \varphi_i \, d\Omega + \sum_{j=1}^{N_h} u_j(t) \int_{\Omega} c^2 \nabla \varphi_j \cdot \nabla \varphi_i \, d\Omega = {G}_i(t),\qquad i=1,..., N_h.
\end{equation}

We can now define the global matrices and semi-discretized vectors:
\begin{itemize}
    \item \textbf{Mass Matrix} $\mathbf{M}$: $M_{ij} = \int_{\Omega} \varphi_j \varphi_i \, d\Omega$.
    \item \textbf{Stiffness Matrix} $\mathbf{A}$: $A_{ij} = \int_{\Omega} c^2 \nabla \varphi_j \cdot \nabla \varphi_i \, d\Omega$.
    \item \textbf{Solution vector} $\mathbf{u}(t) = [u_1(t),\, u_2(t),\, \dots,\, u_{N_h}(t)]^T$
    \item \textbf{RHS} $\mathbf{G}(t) = [G_1(t),\, G_2(t),\, \dots,\, G_{N_h}(t)]^T$.
\end{itemize}

% The assembly process is performed in parallel using \textbf{Trilinos} distributed matrices. For each active cell in the mesh, local contributions are computed using numerical quadrature, specifically the \texttt{QGaussSimplex} rule of order $r+1$ \cite{quarteroni2008numerical}. 

The resulting semi-discrete dynamical system is:
\begin{equation}
\label{eq:semi_discrete_system}
\begin{cases}
\mathbf{M} \, \partial_{tt} \mathbf{u}(t) + \mathbf{A} \mathbf{u}(t) = \mathbf{G}(t) \\
\mathbf{u}(0) = \mathbf{u}_0 \\
\partial_t \mathbf{u}(0) = \mathbf{v}_0
\end{cases}
\end{equation}

This is a second-order Ordinary Differential Equation (ODE) system in time, called the semi-discrete form of the wave equation. 

%In the presence of non-homogeneous Dirichlet conditions, the solver modifies this system during the solution phase to ensure that the degrees of freedom on the boundary correctly reflect the prescribed data $g(\mathbf{x},t)$.

