\documentclass[aspectratio=169]{beamer}

% --- STILE PULITO E MINIMALE ---
\usetheme{Boadilla}
\usefonttheme{professionalfonts}
\setbeamertemplate{navigation symbols}{} % Rimuove i fastidiosi bottoncini in basso

% --- COLORI POLITECNICO DI MILANO ---
\definecolor{PoliBlue}{RGB}{20, 43, 104}
\definecolor{PoliRed}{RGB}{237, 28, 36}

\setbeamercolor{structure}{fg=PoliBlue}
\setbeamercolor{title}{fg=PoliBlue}
\setbeamercolor{frametitle}{fg=PoliBlue}
\setbeamercolor{item}{fg=PoliRed}
\setbeamercolor{block title}{bg=PoliBlue, fg=white}
\setbeamercolor{block body}{bg=PoliBlue!10, fg=black}

\usepackage{amsmath, amssymb, graphicx, booktabs, tikz}

% --- DATI INTESTAZIONE ---
\title[Wave Equation Solvers]{Numerical Solvers for the Wave Equation}
\subtitle{NMPDE project}
\author[Di Lauro, Fabbris, Ghiotto, Rubbi]{R. Di Lauro, T. Fabbris, A. Ghiotto, C. Rubbi}
\institute[PoliMi]{Politecnico di Milano}
\date{February 2026}

\begin{document}

\begin{frame}
    \titlepage
\end{frame}

\begin{frame}{Outline}
    \tableofcontents
\end{frame}

\section{Theoretical framework}
\subsection{Strong and Weak formulation}
\subsection{Spatial Discretization}
\subsection{Time Discretization: \texorpdfstring{$\theta$}{θ} method}
\subsection{Time Discretization: Newmark method}
\section{Results Analysis}
\subsection{Test Problem}
\subsection{Convergence}
\subsection{Dissipation and Dispersion}
\subsection{Scalability}
% \section{Appendix}

% --- SLIDE DI TRANSIZIONE: RESULTS ANALYSIS ---
\begin{frame}[plain, c] % 'plain' rimuove testata/piè di pagina, 'c' centra tutto
    \centering
    \huge \textcolor{PoliBlue}{\textbf{}} \\
    \vspace{0.5cm}
    \Huge \textcolor{PoliBlue}{\textbf{Theoretical framework}}
\end{frame}


\begin{frame}{Strong Formulation \& Spaces}
    \begin{block}{Governing Equation} 
        Let $\Omega \subset \mathbb{R}^2$
    \begin{equation*}
        \begin{cases}
        \partial_{tt} u - c^2 \Delta u = f & \text{in } \Omega \times (0, T] \\
        u = u_0, \quad \partial_t u = v_0 & \text{in } \Omega \text{ at } t=0 \\
        u = \phi & \text{on } \partial\Omega \times (0, T]
        \end{cases}
    \end{equation*}
    \end{block}

    \vspace{0.4cm}
    \textbf{Functional Spaces}
    \begin{itemize}
        \item Solution space: $V_\phi = \{ v \in H^1(\Omega) : v = \phi \text{ on } \partial\Omega \}$
        \item Test space: $V = H^1_0(\Omega) = \{ v \in H^1(\Omega) : v = 0 \text{ on } \partial\Omega \}$
    \end{itemize}
\end{frame}

\begin{frame}{Weak Formulation}
    Let $\mathcal{R}\phi \in V_\phi$ be a lifting operator, then we decompose the solution as $u = \hat{u} + \mathcal{R}\phi$.

    \vspace{0.3cm}
        \begin{block}{Variational Problem}
        Find $\hat{u}(t) \in V$ with $\hat{u}(0) = u_0 - \mathcal{R}\phi(0)$ and $\partial_t \hat{u}(0) = v_0 - \partial_t \mathcal{R}\phi(0)$, such that:
        \begin{equation*}
        {\color{PoliRed} \int_{\Omega} \partial_{tt} \hat{u} \; v \, d\Omega} + a(\hat{u}, v) = G(v) \qquad \forall v \in V
        \end{equation*}
        \end{block}
    \vspace{0.3cm}
    where:
    \begin{itemize}
        \item \textbf{Stiffness form:} $a(\hat{u},v) = \int_{\Omega} c^2 \nabla \hat{u} \cdot \nabla v \, d\Omega$
        \item \textbf{RHS:} $G(v) = \int_{\Omega} f v \, d\Omega - \int_{\Omega} \partial_{tt} \mathcal{R}\phi \; v \, d\Omega - a(\mathcal{R}\phi, v)$
    \end{itemize}
\end{frame}

\begin{frame}{Spatial Discretization}
    \textbf{FEM Approximation:} let $V_{h} \subset V$ with $\dim(V_{h}) = N_h < +\infty$ and $\{\varphi_j\}_{j=1}^{N_h}$ be a basis for $V_h$, then:
    \begin{equation*}
    u_h(\mathbf{x}, t) = \sum_{j=1}^{N_h} u_j(t) \varphi_j(\mathbf{x})
    \end{equation*}
    
    % \vspace{0.1cm}
    \begin{block}{Semi-Discrete Dynamical System}
    \vspace{-0.1cm}
    \begin{equation*}
    \mathbf{M} \, \partial_{tt} \mathbf{u}(t) + \mathbf{A} \, \mathbf{u}(t) = \mathbf{G}(t)
    \end{equation*}
    \end{block}
    
    \vspace{0.2cm}
    \textbf{Algebraic Components:}
    \begin{itemize}
        \item \textbf{Mass matrix:} $M_{ij} = \int_{\Omega} \varphi_j \varphi_i \, d\Omega$
        \item \textbf{Stiffness matrix:} $A_{ij} = \int_{\Omega} c^2 \nabla \varphi_j \cdot \nabla \varphi_i \, d\Omega$
        \item \textbf{Load vector:} $\mathbf{G}_i(t)$ includes forcing and lifting terms.
    \end{itemize}
\end{frame}
\begin{frame}{Time Discretization: $\theta$-Method}
    \textbf{First-Order Reduction:} introduce $\mathbf{v}(t) = \partial_t \mathbf{u}(t)$ and apply a $\theta$-weighted scheme ($\theta \in [0,1]$):
    
    \vspace{0.15cm}
    \begin{block}{$\theta$-Discretization}
    \vspace{-0.2cm}
    \begin{gather*}
    \frac{\mathbf{u}^{n+1} - \mathbf{u}^n}{\Delta t} = \theta\, \mathbf{v}^{n+1} + (1-\theta)\, \mathbf{v}^n \\[4pt]
    \mathbf{M}\, \frac{\mathbf{v}^{n+1} - \mathbf{v}^n}{\Delta t} + \mathbf{A} \bigl( \theta\, \mathbf{u}^{n+1} + (1-\theta)\, \mathbf{u}^n \bigr) = \mathbf{G}_{\theta}^{n+1}
    \end{gather*}
    \end{block}

    \vspace{0.1cm}
    Rearranging and substituting yields the SPD system for $\mathbf{u}^{n+1}$:
    \begin{equation*}
    (\mathbf{M} + \theta^2 \Delta t^2 \mathbf{A})\, \mathbf{u}^{n+1} = \mathbf{M}(\mathbf{u}^n + \Delta t\, \mathbf{v}^n) - \Delta t^2\, \theta(1{-}\theta)\, \mathbf{A}\, \mathbf{u}^n + \Delta t^2\, \theta\, \mathbf{G}_{\theta}^{n+1}
    \end{equation*}

    \vspace{0.1cm}
    At each step we solve \textbf{two SPD systems} sequentially:
    \begin{enumerate}\itemsep0pt
        \item Solve for $\mathbf{u}^{n+1}$ %(RHS depends only on $t^n$ quantities).
        \item Solve $\mathbf{M}\,\mathbf{v}^{n+1} = \ldots$ using the just-computed $\mathbf{u}^{n+1}$.
    \end{enumerate}
    $\theta = 0$: Forward Euler \quad $\theta = 1/2$: Crank--Nicolson \quad $\theta = 1$: Backward Euler
\end{frame}

\begin{frame}{Time Discretization: Newmark Method}
    \textbf{Direct second-order integration} with parameters $\beta$ and $\gamma$. The displacement and velocity are updated via the acceleration $\mathbf{a} \approx \partial_{tt} \mathbf{u}$:
    
    \vspace{0.15cm}
    \begin{block}{Newmark Updates}
    \vspace{-0.2cm}
    \begin{gather*}
    \mathbf{u}^{n+1} = \mathbf{u}^n + \Delta t\, \mathbf{v}^n + \Delta t^2 \bigl[ \bigl(\tfrac{1}{2} - \beta\bigr)\mathbf{a}^n + \beta\, \mathbf{a}^{n+1} \bigr] \\[4pt]
    \mathbf{v}^{n+1} = \mathbf{v}^n + \Delta t \bigl[ (1 - \gamma)\mathbf{a}^n + \gamma\, \mathbf{a}^{n+1} \bigr]
    \end{gather*}
    \end{block}

    \vspace{0.1cm}
    Requiring $\mathbf{M}\mathbf{a}^{n+1} + \mathbf{A}\mathbf{u}^{n+1} = \mathbf{G}^{n+1}$ and substituting yields \textbf{one SPD system}:
    \begin{equation*}
    (\mathbf{M} + \Delta t^2 \beta\, \mathbf{A})\,\mathbf{a}^{n+1} = \mathbf{G}^{n+1} - \mathbf{A}\mathbf{u}^n - \Delta t\, \mathbf{A}\mathbf{v}^n - \Delta t^2 \bigl(\tfrac{1}{2} - \beta\bigr) \mathbf{A}\mathbf{a}^n
    \end{equation*}

    \vspace{0.1cm}
    Then $\mathbf{u}^{n+1}$ and $\mathbf{v}^{n+1}$ are updated algebraically. Notable choices:
    \begin{itemize}\itemsep0pt
        \item $\gamma = 1/2,\; \beta = 1/4$: \textbf{Average Acceleration} %(unconditionally stable, non-dissipative).
        \item $\gamma = 1/2,\; \beta = 0$: \textbf{Central Difference} %(explicit, conditionally stable).
    \end{itemize}
\end{frame}



% ==========================================
% SECTION 3: RESULTS ANALYSIS
% ==========================================

% --- SLIDE DI TRANSIZIONE: RESULTS ANALYSIS ---
\begin{frame}[plain, c] % 'plain' rimuove testata/piè di pagina, 'c' centra tutto
    \centering
    \huge \textcolor{PoliBlue}{\textbf{}} \\
    \vspace{0.5cm}
    \Huge \textcolor{PoliBlue}{\textbf{Results Analysis}}
\end{frame}

\begin{frame}{Solution Examples}
    Three simulations with different problem data:
    \vspace{0.3cm}
    \begin{columns}[T]
        \column{0.33\textwidth}
        \centering
        \includegraphics[width=\textwidth]{images/sol-square.png}\\[4pt]
        {\small Square wave}
        \column{0.33\textwidth}
        \centering
        \includegraphics[width=\textwidth]{images/sol-standing.png}\\[4pt]
        {\small Standing wave}
        \column{0.33\textwidth}
        \centering
        \includegraphics[width=\textwidth]{images/sol-traveling.png}\\[4pt]
        {\small Traveling wave}
    \end{columns}

    \vspace{0.5cm}
    Videos of the simulations are available in this \href{https://www.youtube.com/playlist?list=PLh9fwY76Oobay4rXdlAKkHKuC61a1lL61}{YouTube playlist}
\end{frame}

\begin{frame}{Test Problem}
    \textbf{Domain:} Unit square $\Omega = [0,1]^2$, final time $T = 1$.

    \begin{block}{Standing-Mode Wave}
    \vspace{-0.1cm}
    \begin{equation*}
    \begin{cases}
    \partial_{tt} u - \Delta u = 0 & \text{in } \Omega \times (0, T] \\[6pt]
    u = 0 & \text{on } \partial\Omega \times [0, T] \\[6pt]
    u(x, y, 0) = \sin(\pi x)\sin(\pi y) & \text{in } \Omega \\[6pt]
    \partial_t u(x, y, 0) = 0 & \text{in } \Omega
    \end{cases}
    \end{equation*}
    \end{block}


\vspace{0.3cm}
    \textbf{Exact Analytical Solution:}
    \begin{equation*}
    u_{\text{exact}}(x, y, t) = \cos(\sqrt{2}\pi t) \sin(\pi x)\sin(\pi y)
    \end{equation*}
    
    \vspace{0.1cm}
    \textbf{Exact Angular Frequency:} 
    \begin{equation*}
    \omega_{\text{exact}} = \sqrt{2}\pi
    \end{equation*}
    
\end{frame}


\begin{frame}[plain, c]
    \centering
    \huge \textcolor{PoliBlue}{\textbf{}} \\
    \vspace{0.5cm}
    \Huge \textcolor{PoliBlue}{\textbf{Convergence}} \\
    \vspace{0.8cm}
    
\end{frame}

\begin{frame}{Convergence Analysis: Sweep Parameters}
    \textbf{Parameter Grid (490 total runs)}
    \begin{table}
        \centering
        \begin{tabular}{ll}
        \toprule
        \textbf{Parameter} & \textbf{Values} \\
        \midrule
        Elements per side $N_{\text{el}}$ & 10, 20, 40, 80, 160, 320 \\
        FE polynomial degree $r$ & 1, 2 \\
        Time step $\Delta t$ & $10^{-1}, 5 \times 10^{-2}, \dots, 10^{-4}$ (10 values) \\
        \bottomrule
        \end{tabular}
    \end{table}

    \textbf{Tested Time-Stepping Schemes}
    \begin{table}
        \centering
        \begin{tabular}{llcc}
        \toprule
        \textbf{Scheme} & \textbf{Parameters} & \textbf{Accuracy} & \textbf{Stability} \\
        \midrule
        $\theta$-method (FE) & $\theta = 0$ & $\mathcal{O}(\Delta t)$ & Conditional \\
        $\theta$-method (CN) & $\theta = 0.5$ & $\mathcal{O}(\Delta t^2)$ & Unconditional \\
        $\theta$-method (BE) & $\theta = 1$ & $\mathcal{O}(\Delta t)$ & Unconditional \\
        Newmark (explicit) & $\beta=0,\; \gamma=0.5$ & $\mathcal{O}(\Delta t^2)$ & Conditional \\
        Newmark (implicit) & $\beta=0.25,\; \gamma=0.5$ & $\mathcal{O}(\Delta t^2)$ & Unconditional \\
        \bottomrule
        \end{tabular}
    \end{table}

    % \begin{itemize}
    %     \item \textbf{CFL Compliance:} Explicit schemes were automatically filtered using a safety factor of $0.9$.
    % \end{itemize}
\end{frame}

\begin{frame}{Convergence Analysis: A Priori Estimate}
    \begin{block}{Theoretical Error Bound}
    The total numerical error at time $t_n$ is bounded by the sum of spatial and temporal contributions:
    \begin{equation*}
    \|u(\cdot,t_n) - u_h^n\| \;\leq\; C\bigl(h^{s} + \Delta t^{q}\bigr)
    \end{equation*}
    \end{block}
    
    \vspace{0.4cm}
    \textbf{Expected Convergence Rates:}
    \begin{itemize}
        \item \textbf{Spatial order ($s$):} 
        \begin{itemize}
            \item $s = r+1$ for the $L^2$-norm error.
            \item $s = r$ for the $H^1$-norm error.
        \end{itemize}
        \item \textbf{Temporal order ($q$):} $q \in \{1, 2\}$, as defined for each scheme.
    \end{itemize}
    \end{frame}
    
\begin{frame}{BE ($\theta = 1$): $\mathcal{O}(\Delta t)$}
    \centering
    
    \includegraphics[width=0.48\textwidth]{images/conv_theta_1_dt_r1_2_L2.png}
    \hfill
    \includegraphics[width=0.48\textwidth]{images/conv_theta_1_h_r1_2_L2.png}

    \vspace{0.3cm}

    \begin{itemize}
        \item \textbf{Plateau Effect:} For $r=2$, the $L^2$ error is bounded by $C(h^3 + \Delta t)$.
        \item The 1st-order temporal error dominates as $h \to 0$.
    \end{itemize}
\end{frame}

\begin{frame}{Implicit Newmark ($\beta=0.25$) / CN ($\theta = 0.5$): $\mathcal{O}(\Delta t^2)$}
    \centering
    \includegraphics[width=0.48\textwidth]{images/conv_newmark_025_dt_r1_2_L2.png}
    \hfill
    \includegraphics[width=0.48\textwidth]{images/conv_newmark_025_h_r1_2_L2.png} 

    \vspace{0.3cm}
    \begin{itemize}
        \item \textbf{Improved Accuracy:} The error is bounded by $C(h^s + \Delta t^2)$.
        \item Now the bound imposed by $\Delta t$ improves, so the curves in the "Error vs h" are less flat.
    \end{itemize}
\end{frame}

\begin{frame}{Explicit Newmark ($\beta=0$) / FE ($\theta = 0$): conditional stability}
    \centering
    \includegraphics[width=0.48\textwidth]{images/conv_newmark_000_dt_r1_2_L2.png}
    \hfill
    \includegraphics[width=0.48\textwidth]{images/conv_newmark_000_h_r1_2_L2.png} 

    \vspace{0.3cm}
    \begin{itemize}
        \item \textbf{CFL condition:} only configurations satisfying $\Delta t \le C \frac{h}{c}$ are stable in time.
        \item If $\Delta t$ is too large compared to $h$, the numerical solution diverges.
        \item This behavior is visible in the plots:
        \begin{itemize}
            \item In the convergence plot w.r.t $\Delta t$, data points are missing on the RHS (large $\Delta t$).
            \item In the convergence plot w.r.t $h$, data points are missing on the LHS (small $h$).
        \end{itemize}
    \end{itemize}
\end{frame}


\begin{frame}[plain, c]
    \centering
    \huge \textcolor{PoliBlue}{\textbf{}} \\
    \vspace{0.5cm}
    \Huge \textcolor{PoliBlue}{\textbf{Dissipation and Dispersion}} \\
    \vspace{0.8cm}
    
\end{frame}

\begin{frame}{Numerical Dissipation: Energy Evolution}
    \centering
    % Carica il file nella colonna a sinistra per risolvere l'errore "File not found"
    \includegraphics[width=0.9\textwidth]{images/dissdisp_energy_evolution.png} 
    
    \vspace{0.2cm}
    The exact solution conserves energy, so the reference is $E(t)/E(0) = 1$.
    \begin{itemize}
        \item \textbf{Setup:} $T=5$ s , $N_{el}=60$, $r=1$.
        \item \textbf{Energy-Conserving:} \textbf{CN} and \textbf{Implicit Newmark} preserve $E(t)/E(0) = 1$.
        \item \textbf{Dissipative:} \textbf{BE} exhibits strong decay. For large $\Delta t$, the wave is almost entirely damped.
        \item \textbf{Stability:} \textbf{FE} and \textbf{Explicit Newmark} are conservative within the CFL bound.
    \end{itemize}
\end{frame}



\begin{frame}{Numerical Dispersion: Point-Probe Analysis}
    % \vspace{0.2cm}
    \begin{columns}[T]
        \column{0.55\textwidth}
        \centering
        \includegraphics[width=0.9\textwidth]{images/dissdisp_probe_vs_exact.png}

        \column{0.50\textwidth}
        \textbf{Central Probe:} $u(0.5, 0.5, t) = \cos(\sqrt{2}\pi t)$ \\ 
        Comparison of phase drift $\tilde{\omega} \neq \omega_{\text{exact}}$.
        \begin{itemize}\itemsep6pt
            \item \textbf{CN / Implicit Newmark:} No dissipation, but visible \textbf{phase lag} for large $\Delta t$.
            \item \textbf{BE:} Highly dissipative + phase lag, unsuitable for long-time propagation.
            \item \textbf{Explicit schemes:} when it doesn't diverge, it overlaps perfectly with the exact solution.
        \end{itemize}
    \end{columns}
\end{frame}

%%%%%%%%%%%%%%%%%%%%%%%%%%%%%%%%%%%%%%%%%%%%%%%%%%%%%%%%%%%%%%%%%%%%%%%%%%%%%%%%%%%%%%%%%%%%%
\begin{frame}{Spectral Analysis}
    \centering
    % Limitiamo l'altezza dell'immagine per salvare spazio verticale
    \includegraphics[width=\textwidth, height=0.55\textheight, keepaspectratio]{images/sinc_good.png} 
    
    \vspace{0.1cm}
    
    % Setup compatto
    {\footnotesize \textbf{Test Setup:}  $\Delta t = 0.15$ s}
    \vspace{0.1cm}
    
    \begin{columns}[T]
        % --- Colonna Sinistra (CN e NM) ---
        \column{0.48\textwidth}
        \footnotesize 
        \textbf{CN ($\theta=0.5$), Newmark ($\beta=0.25, \gamma=0.5$)}
        \begin{itemize}\itemsep0pt 
            % \item Both schemes preserve the high spectral amplitude ($3. \times 10^{-1}$).
            \item \textbf{Frequency error:} $\Delta f = 0.024$ Hz
        \end{itemize}
        
        % --- Colonna Destra (BE) ---
        \column{0.48\textwidth}
        \footnotesize
        \textbf{Backward Euler ($\theta=1.0$)}
        \begin{itemize}\itemsep0pt
            % \item \textbf{Amplitude Collapse:} Drops by over an order of magnitude ($2.35 \times 10^{-2}$).
            \item \textbf{Frequency error:} $\Delta f = 0.050$ Hz
        \end{itemize}
    \end{columns}
\end{frame}
%%%%%%%%%%%%%%%%%%%%%%%%%%%%%%%%%%%%%%%%%%%%%%%%%%%%%%%%%%%%%%5

\begin{frame}[plain, c]
    \centering
    \huge \textcolor{PoliBlue}{\textbf{}} \\
    \vspace{0.5cm}
    \Huge \textcolor{PoliBlue}{\textbf{Scalability}} \\
    \vspace{0.8cm}
    
\end{frame}

\begin{frame}{Strong Scaling Analysis}
    \centering
    \includegraphics[width=0.85\textwidth]{images/scalability_strong_scaling.png} 
    
    \vspace{0.1cm}
    \begin{itemize}
        \item \textbf{Setup:} $N_{el} = 640$ ($>400k$ DoFs), $\Delta t = 8 \cdot 10^{-5}$, $T=0.05$ (625 steps).
        \item \textbf{Cluster Environment:} up to 16 MPI ranks on Intel Xeon Gold 6238R.
        \item \textbf{Precision:} Processes pinned to physical cores (\texttt{--bind-to core}) with hyper-threading disabled (\texttt{OMP\_NUM\_THREADS=1}).
        \item \textbf{I/O Optimization:} Execution on \texttt{/scratch\_local} for fast node-local storage.
        % \item \textbf{Results:} Speedup $S(16) \approx 11\times$ with parallel efficiency $E(16) \in [66\%, 72\%]$.
    \end{itemize}
\end{frame}

\begin{frame}{Computational Efficiency: $\theta$-methods vs. Newmark}
    \begin{table}
        \footnotesize
        \centering
        \begin{tabular}{lcccc}
            \toprule
            \textbf{Scheme} & $T(1)$ [s] & $T(16)$ [s] & $S(16)$ & $E(16)$ \\ \midrule
            $\theta$-methods (FE, BE, CN) & $\sim 645$ & $\sim 56$ & $\approx 11.4$ & $72\%$ \\
            Newmark ($\beta \in \{0, 0.25\}$) & $\sim 313$ & $\sim 29$ & $\approx 10.6$ & $67\%$ \\ \bottomrule
        \end{tabular}
        \caption{Strong scaling summary at $p=16$ MPI processes}
    \end{table}

    \begin{itemize}
        \item \textbf{Throughput:} Newmark schemes are \textbf{twice as fast} as $\theta$-methods.
        \item \textbf{Linear Solves:} $\theta$-methods solve two SPD systems per step ($\mathbf{u}^{n+1}, \mathbf{v}^{n+1}$); Newmark solves only one for $\mathbf{a}^{n+1}$.
        \item \textbf{Parallel Trade-off:} Newmark's lower per-step cost makes non-parallelizable overhead more prominent, leading to slightly lower efficiency.
    \end{itemize}
\end{frame}




\begin{frame}[plain, c] % 'plain' rimuove testata/piè di pagina, 'c' centra tutto
    \centering
    \huge \textcolor{PoliBlue}{\textbf{}} \\
    \vspace{0.5cm}
    \Huge \textcolor{PoliBlue}{\textbf{Thanks for you attention!}}
\end{frame}


% SLIDES BONUS

\begin{frame}[noframenumbering]{Explicit Newmark ($\beta=0$)}
    \centering
    \includegraphics[width=0.9\textwidth]{images/conv_newmark_025_dt_r1_2.png}
    \includegraphics[width=0.9\textwidth]{images/conv_newmark_025_h_r1_2.png} 
\end{frame}

\begin{frame}[noframenumbering]{Implicit Newmark ($\beta=0.25$)}
    \centering
    \includegraphics[width=0.9\textwidth]{images/conv_newmark_000_dt_r1_2.png}
    \includegraphics[width=0.9\textwidth]{images/conv_newmark_000_h_r1_2.png} 
\end{frame}

\begin{frame}[noframenumbering]{BE ($\theta=1.0$)}
    \centering
    \includegraphics[width=0.9\textwidth]{images/conv_theta_1_dt_r1_2.png}
    \includegraphics[width=0.9\textwidth]{images/conv_theta_1_h_r1_2.png} 
\end{frame}

\begin{frame}[noframenumbering]{CN ($\theta=0.5$)}
    \centering
    \includegraphics[width=0.9\textwidth]{images/conv_theta_05_dt_r1_2.png}
    \includegraphics[width=0.9\textwidth]{images/conv_theta_05_h_r1_2.png} 
\end{frame}

\begin{frame}[noframenumbering]{FE ($\theta=0.0$)}
    \centering
    \includegraphics[width=0.9\textwidth]{images/conv_theta_0_dt_r1_2.png}
    \includegraphics[width=0.9\textwidth]{images/conv_theta_0_h_r1_2.png} 
\end{frame}


\begin{frame}[noframenumbering]{Spectral Analysis: 5-modes Case}
    \centering
    
    % Soluzione esatta completa e Setup (fusi in un blocco compatto)
    {\tiny $u(x,y,0) = 0.2s_{11} + 0.15s_{21} + 0.1s_{12} + 0.08s_{22} + 0.05s_{31} \quad [s_{nm} = \sin(n\pi x)\sin(m\pi y)]$} \\
    {\scriptsize \textbf{Solution:} $u = \sum A_{nm} \cos(\omega_{nm} t) \sin(n\pi x) \sin(m\pi y)$, \qquad \textbf{Setup:} $\Delta t = 0.15$s, $P(0.25, 0.25)$}.
    
    \vspace{0.1cm}
    
    \includegraphics[width=0.95\textwidth, height=0.55\textheight, keepaspectratio]{images/sinc_5.png} 
    
    \vspace{0.2cm}
    
    \begin{columns}[T]
        \column{0.48\textwidth}
        \footnotesize 
        \textbf{Conservative (CN / Newmark)}
        \begin{itemize}\itemsep2pt 
            \item \textbf{Full Spectrum:} All harmonics preserved.
            \item \textbf{Dispersion:} Error $2.9\% \to 3.8\%$.
        \end{itemize}
        
        \column{0.48\textwidth}
        \footnotesize
        \textbf{Dissipative (Backward Euler)}
        \begin{itemize}\itemsep2pt
            \item \textbf{Numerical LPF:} High modes obliterated.
            \item \textbf{Decay:} Amplitude collapse $>95\%$.
        \end{itemize}
    \end{columns}
\end{frame}

\begin{frame}[noframenumbering]{Spectral Analysis: 5-modes Case}
    \centering
    
    % Intestazione minimalista
    {\scriptsize \textbf{Solution:} $u = \sum A_{nm} \cos(\omega_{nm} t) \sin(n\pi x) \sin(m\pi y)$ \quad $\bullet$ \quad \textbf{Setup:} $\Delta t = 0.15$s, $P(0.25, 0.25)$}
    
    \vspace{0.2cm}
    
    \includegraphics[width=0.95\textwidth, height=0.55\textheight, keepaspectratio]{images/sinc_5.png} 
    
    \vspace{0.3cm}
    
    \begin{columns}[T]
        \column{0.48\textwidth}
        \footnotesize 
        \textbf{Conservative (CN, Newmark)}
        \begin{itemize}\itemsep2pt 
            \item Energy preserved across all modes.
            \item Frequency error: $2.9\% \to 3.8\%$ (dispersion).
        \end{itemize}
        
        \column{0.48\textwidth}
        \footnotesize
        \textbf{Dissipative (Backward Euler)}
        \begin{itemize}\itemsep2pt
            \item Numerical LPF: higher modes obliterated.
            \item Amplitude collapse ($>95\%$).
        \end{itemize}
    \end{columns}
\end{frame}


\begin{frame}[noframenumbering]{Scalability 32}
    \centering
    \includegraphics[width=\textwidth]{images/scalability_strong_scaling_32.png}
\end{frame}


% \begin{frame}{Spatial Convergence: Forward Euler ($\theta = 0$)}
%     \centering
%     \includegraphics[width=0.85\textwidth]{images/conv_theta_0_h_r1_2.png}
    
%     \vspace{0.3cm}
%     \begin{itemize}
%         \item \textbf{CFL Constraint:} Only stable combinations of $(h, \Delta t)$ are shown; the sweep script automatically filtered out unstable runs.
%         \item \textbf{1st Order Accuracy:} Similar to BE, the spatial convergence is limited by the $\mathcal{O}(\Delta t)$ temporal error.
%     \end{itemize}
% \end{frame}

% \begin{frame}{Spatial Convergence: Newmark Explicit ($\beta = 0$)}
%     \centering
%     \includegraphics[width=0.85\textwidth]{images/conv_newmark_000_h_r1_2.png}
    
%     \vspace{0.3cm}
%     \begin{itemize}
%         \item \textbf{Second Order:} As a 2nd-order method in time, it allows for deeper spatial convergence before reaching the plateau.
%         \item \textbf{Stability:} Conditionally stable, requires adherence to the CFL condition.
%     \end{itemize}
% \end{frame}

% \begin{frame}{Spatial Convergence: Newmark Implicit ($\beta = 0.25$)}
%     \centering
%     \includegraphics[width=0.85\textwidth]{images/conv_newmark_025_h_r1_2.png}
    
%     \vspace{0.3cm}
%     \begin{itemize}
%         \item \textbf{Unconditional Stability:} Combines the 2nd-order accuracy of CN with robust stability.
%         \item \textbf{High-Order Synergy:} Perfectly recovers $\mathcal{O}(h^3)$ for $r=2$ when $\Delta t$ is sufficiently small.
%     \end{itemize}
% \end{frame}


% \begin{frame}{Temporal Convergence: Forward Euler ($\theta = 0$)}
%     \centering
%     \includegraphics[width=0.85\textwidth]{images/conv_theta_0_dt_r1_2.png}
    
%     \vspace{0.3cm}
%     \begin{itemize}
%         \item \textbf{CFL Limitation:} Only the stable portion of the sweep is visible; the scheme confirms $\mathcal{O}(\Delta t)$ accuracy within its stability region.
%         \item \textbf{Comparison:} Displays similar accuracy to Backward Euler but requires strictly small time steps to avoid divergence.
%     \end{itemize}
% \end{frame}

% \begin{frame}{Temporal Convergence: Newmark Explicit}
%     \centering
%     \includegraphics[width=0.85\textwidth]{images/conv_newmark_000_dt_r1_2.png}
    
%     \vspace{0.3cm}
%     \begin{itemize}
%         \item \textbf{Second-Order Accuracy:} Confirms $\mathcal{O}(\Delta t^2)$ rate, providing a significant accuracy boost over the $\theta$-method (except for CN).
%         \item \textbf{Conditional Stability:} Stability is governed by the mesh size, requiring the automated filtering of the parameter grid.
%     \end{itemize}
% \end{frame}


% \begin{frame}{Temporal Convergence: Newmark Implicit}
%     \centering
%     \includegraphics[width=0.85\textwidth]{images/conv_newmark_025_dt_r1_2.png}
    
%     \vspace{0.3cm}
%     \begin{itemize}
%         \item \textbf{Robustness:} Combines unconditional stability with second-order temporal accuracy $\mathcal{O}(\Delta t^2)$.
%         \item \textbf{Optimal Performance:} Exhibits the best balance between stability and error reduction, especially for high-order spatial elements ($r=2$).
%     \end{itemize}
% \end{frame}



\end{document}
